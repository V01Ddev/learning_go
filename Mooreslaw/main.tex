\documentclass[12pt]{article}

\title{Moore’s law}
\author{Mohammed Aljomaily}
\date{\today}

\begin{document}

\maketitle

\section{What is Moore’s Law?}
In 1975, it was observed that the number of transistors in an integrated circuit (IC) doubles approximately every two years. This exponential growth has driven rapid advancements in computing power, enabling more powerful and compact electronic devices over time.

\section{Limitation of Moore's Law}

\subsection*{Physical Limits of Transistors}
As transistors approach atomic scales (around 1 to 2 nanometers), they encounter fundamental physical limitations. Quantum tunneling becomes significant at these scales, allowing electrons to pass through barriers that should be insulators, leading to leakage currents and unreliable behavior. These quantum effects challenge the continued miniaturization of transistors.

\subsection*{Power and Efficacy Issues}
Increasing transistor density leads to higher power consumption. As more transistors are packed into a chip, the power density rises, which in turn causes temperature to increase. This excess heat becomes increasingly difficult to dissipate, potentially leading to thermal management issues and reduced performance or reliability.

\subsection*{Voltage Scaling Limitations}
Voltage scaling has traditionally been used to reduce dynamic power consumption in integrated circuits. However, as voltage levels decrease, leakage power becomes a significant concern that voltage scaling alone cannot mitigate. Additionally, further voltage reduction is constrained by noise margins and threshold voltage requirements, limiting its effectiveness.

\subsection*{Logistical and Cost Issues}
The cost of materials and manufacturing has escalated due to the increasing complexity and precision required for smaller transistors. As components shrink, the fabrication process becomes more intricate and expensive, making it economically challenging to sustain the pace predicted by Moore’s Law.

\end{document}
